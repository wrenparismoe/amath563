\documentclass[12pt]{article}
%\usepackage{xcolor}
\usepackage[top=2cm, bottom=2cm, left=2cm, right=2cm, headsep=2mm, foot=4mm]{geometry}
\usepackage{amsmath,amsthm,amsfonts,amssymb,amscd}
\usepackage{enumitem}
\usepackage{parskip}
\usepackage{mathtools}
\usepackage{mathrsfs}
\usepackage{mdframed}
%\usepackage{graphicx}
%\usepackage{hyperref}
%\graphicspath{ {./img/} }

%\newcommand{text}[1]{\makebox[0pt][l]{#1}}
\DeclareMathOperator{\dom}{dom}
\DeclareMathOperator{\range}{range}

\title{Homework 1 - AMATH 583}
\author{Warren Paris-Moe}
\date{April 2023}

\begin{document}

\section*{THEORY}

\subsection*{Problem 1}
\begin{mdframed}
    Suppose $\Gamma: \mathcal{X} \times \mathcal{X} \to \mathbb{R}$ is a PDS kernel. 
    Prove that $\forall x, x' \in \mathcal{X}$ it holds that
    $|\Gamma(x, x')|^2 \leq \Gamma(x, x) \Gamma(x', x')$.
\end{mdframed}

Let $\Gamma:\mathcal{X}\times\mathcal{X}arrow\mathbb{R}$ be a positive definite symmetric (PDS)
kernel in the real space and $x,x' \in \mathcal{X}$. Say $x=x_1$ and $x'=x_2$. Then, since $\Gamma$ 
is PDS, we know that the $2 \times 2$ kernel matrix $K$ with entries $K_{ij} = \Gamma(x_i, x_j)$ is 
positive definite. Therefore, its eigenvalues are positive. Thus, it follows that the product of 
those eigenvalues, aka the determinant of $K$, is also positive. That is,
\begin{align*}
    0 \leq K_{11}K_{22} - K_{12}K_{21} &= K_{11}K_{22} - K_{12}K_{12}^* \\
    &= K_{11}K_{22} - |K_{12}|^2 \\
    &= \Gamma(x_1, x_1)\Gamma(x_2, x_2) - |\Gamma(x_1, x_2)|^2 \\
    &= \Gamma(x, x)\Gamma(x', x') - |\Gamma(x, x')|^2
\end{align*}
where $K_{12}^*$ is the complex conjugate of $K_{12}$. Since $\Gamma$ is real-valued,
$K_{12}^* = K_{12}$. Thus, we have that 
\begin{align}
    |\Gamma(x, x')|^2 \leq \Gamma(x, x)\Gamma(x', x')
\end{align}


\subsection*{Problem 2} % Problem 2
\begin{mdframed}
    Given a kernel $K$ on $\mathcal{X}$ define its normalized version as
    \[
    \bar{K}(x,x') = 
        \begin{cases}
            0 & \text{if } K(x,x)=0 \text{ or } K(x', x')=0 \\ 
            \frac{K(x, x')}{\sqrt{K(x, x)} \sqrt{K(x', x')}} & \text{Otherwise.} 
        \end{cases}
    \]
    Show that if $K$ is PDS then so is $\bar{K}$.
\end{mdframed}


\subsection*{Problem 3} % Problem 3
\begin{mdframed}
    Show that the following kernels on $\mathbb{R}^{d}$ are PDS:
    \begin{itemize}
        \item Polynomial kernel: $K(x, x')=(x^{T} x'+c)^{\alpha}$ for $c>0$ and $\alpha \in \mathbb{N}$.
        \item Exponential kernel: $K(x, x')=\exp (x^Tx')$.
        \item RBF kernel: $K(x, x')=\exp (-\gamma^{2}\|x-x'\|_2^2)$.
    \end{itemize}
\end{mdframed}


\subsection*{Problem 4} % Problem 4
\begin{mdframed}
    Let $\Omega \subseteq \mathbb{R}^{d}$ and let $\{\psi_j\}_{j=1}^n$ be a sequence of continuous 
    functions on $\Omega$ and $\{\lambda_j\}_{j=1}^n$ a sequence of non-negative numbers. Show that 
    $K(x,x')=\sum_{j=1}^n \lambda_j\psi_j(x)\psi_j(x')$ is a $\operatorname{PDS}$ kernel on $\Omega$.
\end{mdframed}



\subsection*{Problem 5} % Problem 5
\begin{mdframed}
    Show that:
    \begin{enumerate}[topsep=0pt, partopsep=0pt, itemsep=0pt, label=(\roman*)]
        \item if $K$ and $K'$ are two reproducing kernels for an RKHS $\mathcal{H}$, 
            then they have to be the same. \\
        \item the RKHS of a PDS kernel $K$ is unique.
    \end{enumerate}
\end{mdframed}





\section*{COMPUTATION}

\subsection*{EXPERIMENTS}


\end{document}